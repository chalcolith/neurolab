\documentclass[12pt,letterpaper,twocolumn]{article}
\usepackage[utf8x]{inputenc}
\usepackage{ucs}
\usepackage{amsmath}
\usepackage{amsfonts}
\usepackage{amssymb}
\usepackage{lingmacros}
\usepackage{tree-dvips}
\usepackage{tipa}
\usepackage[numbers,sort&compress]{natbib}

\usepackage{ifthen}
\usepackage{color}

% If set to true, comments will be inserted, otherwise ignored
\newboolean{showComments}
\setboolean{showComments}{true}

% Internal command
\newcommand{\todocmd}[1]{{\bf\textcolor{red}{#1}}}
 
% use \todo{text} for a comment within your page/section
\newcommand{\todo}[1]{\ifthenelse {\boolean{showComments}} {\todocmd{#1}} {}}
 
% use \todof{text} for a footnote
\newcommand{\todof}[1]{\ifthenelse {\boolean{showComments}} {\footnote{\todocmd{#1}}} {}}

%%%%%%%%%%%%%%%%%%%%%%%%%%%%%%%%%%%%%%%%%%%%%%%%%%%%%%%%%%%%%%%%%%%%%%%%%%%%%%%%%%%%%%%%%%%%%

\begin{document}
\author{Gordon Tisher\\gordon\@balafon.net}
\title{Some Experiments in Neurocognitive Linguistics}
\maketitle

\begin{abstract}
Blah blah blah
\end{abstract}

%%%%%%%%%%%%%%%%%%%%%%%%%%%%%%%%%%%%%%%

\section{Introduction}

% various applications here:

% lamb's examples (from L&R ch15, class notes) -- simple syntactic structures

% Topicalization simply replicates the order of activation in the brain.  Passive, frex, is simply the brain thinking of something, when you realize that it is the patient, you choose (or "is activated") the passive template rather than the active.  No transformation :-)

\section{Implementation}

% Ways of implementing:

% - usual neural net method of matrices (what about Duch!)
% - cellular automaton -- as compared to Duch, who tries to abstract the spreading activations, we will use brute force

% Using \citet{Nehaniv2002A, Nehaniv2004A}'s technique for generating an asynchronous cellular automaton.

% examples with screenshots

\section{Categories}

% The relational neural network model is uninteresting inasmuch as it describes individual utterances.  The really interesting part is when we get to variable lexemes.  Eventually, all grammar is overlapping sets of variable lexemes.

% All grammar is a collection of parameterized idioms (a la lamb chapter 17).
% How to learn these?


%%%%%%%%%%%%%%%%%%%%%%%%%%%%%%%%%%%%%%%%%%%%%%%%%%%%%%%%%%%%%%%%%%%%%%%%%%%%%%%%%%%%%%%%%%%%%%

\bibliographystyle{plainnat}
\bibliography{references}
\end{document}
